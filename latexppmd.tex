%----------------------------------------------------------------
% FEUILLE DE STYLE ENSG au format Latex
%Classe de document pour le thème Latex de l'ENSG
% v.1 sept. 2010, D Lercier : création
% v.2 sept. 2012, T Coupin : création fichier classe et modif.
% v.3 sept. 2016, J. Beilin, modification de la gestion de la biblio + modifs mineures 
%----------------------------------------------------------------

\documentclass{themeensg}
\usepackage{color}

%---Texte en filigranne---
\SetWatermarkText{\textsc{Brouillon}}
%pour l'enlever : \SetWatermarkText{}
\SetWatermarkText{}

% mise en page no-stress
\renewcommand{\familydefault}{\sfdefault}
%-------------------------

%---Mes packages à moi---
%\usepackage{}
%------------------------

%---Mes raccourcis---
\newcommand{\transpose}[1]{{}^t \! #1}
\newcommand{\ensg}{\textsc{Ensg}}

\renewcommand{\author}{Bob L'Éponge}

%--------s------------

%---Paramètres du pdf---
    \hypersetup{
       backref=true,                           % Permet d'ajouter des liens dans
       pagebackref=true,                       % les bibliographies
       hyperindex=true,                        % Ajoute des liens dans les index.
       colorlinks=true,  %Colorise les liens : true pour version numérique, false pour version d'impression
       breaklinks=true,                        % Permet le retour à la ligne dans les liens trop longs.
       urlcolor= blue,                         % Couleur des hyperliens.
       linkcolor= blue,                       % Couleur des liens internes.
       bookmarks=true,                         % Créé des signets pour Acrobat.
       %bookmarksopen=true,                    % Si les signets Acrobat sont créés,
                                               % les afficher complètement.
       pdftitle={Thème ENSG},                 % Titre du document.
                                               % Informations apparaissant dans
       pdfauthor={\author},                      % dans les informations du document
       pdfsubject={Feuille de style ENSG}           % sous Acrobat.
    }

%-----------------------



%-------------------------------------------------------------

\setcounter{tocdepth}{1} %profondeur de la table des matières

\title{Feuille de style \LaTeX~de l'\ensg\\Documentation plus que rapide\\Version provisoire du \today~à \timenow}

\bibliography{latexppmd}

%
%-------------------------------------------------------------
% Début du document
%--------------------------------------------------------------
\begin{document}
%--------------------------------------------------------------
\begin{titlepage}
%Inclusion des labels des entreprises
%Pour un seul label (à gauche), mettre NULL pour les 3e et 4e argument
\enterprise 
{logos/logo_ensg}
{Ecole Nationale des Sciences Géographiques}
{logos/logo_entreprise}
{Nom de l'entreprise d'accueil}

%Inclusion du titre

\textit{{\color{red}Choisir les intitulés et responsables pédagogiques (p2) qui conviennent suivant votre cursus - ing3 ou MS PPMD-}}

\maketitle{{\color{red}Rapport de stage} ou {\color{magenta}Thèse Professionnelle}\\{\color{red}Cycle des Ingénieurs diplômés de l'ENSG 3\up{ème} année} ou {\color{magenta}Mastère spécialisé$\circledR$ Photogrammétrie, Positionnement, Mesure de Déformations}}{logos/logo_ensg}



\infos{\author}{Novembre 2017}
\end{titlepage}

%\begin{comment}
% ---Page du jury---
%---Page du jury---
%\newevenpage
\thispagestyle{plain}
\section*{Jury}
\vspace{0.5cm}

\textbf{Président de jury :} \\

Le président de jury

\vspace{0.5cm}

\textbf{Commanditaire :} \\

le commanditaire

\vspace{0.5cm}

\textbf{Encadrement de stage :} \\ 


qui a encadré ?

\vspace{0.5cm}

\textbf{Enseignant référent :} \\ 

qui a assuré le suivi pédagogique côté ENSG ?

\vspace{0.5cm}

\textbf{Rapporteur expert :} \\ 

qui est rapporteur du mémoire ?

\vspace{0.5cm}

\textbf{Responsable{\color{magenta}s} pédagogique{\color{magenta}s} du {\color{red} cycle Ingénieur} ou du {\color{magenta}MS$\circledR$ PPMD} :} \\



{\color{red}Jean-François Hangouët, IGN/ENSG/PEGI}

ou 

{\color{magenta}
Jacques Beilin, IGN/ENSG/PEGMD

Antoine Pinte, IGN/ENSG/DFI
}

\vspace{0.5cm}

\textbf{Gestion du stage :} \\ 

Delphine Genes, Claire Driessens, IGN/ENSG/DFI

\vspace{0.5cm}


\copyright \hspace{0.3cm} ENSG

\section*{Stage de fin d'étude du xxx au xxx }
\vspace{0.3cm}
\textbf{Diffusion web :} $\boxtimes$ Internet \hspace{0.2cm}$\boxtimes$ Intranet\hspace{0.2cm}
\vspace{0.3cm}

\textbf{Situation du document :} 
\vspace{0.2cm}
\par
{\color{red}Rapport de stage de fin d'études présenté en fin de 3\up{ème} année du cycle des Ingénieurs}

ou

{\color{magenta}Thèse professionelle présenté en vue de l'obtention du MS$\circledR$ PPMD}
\vspace{0.3cm}


\newcounter{x}
\setcounter{x}{\getpagerefnumber{LastPage}-\getpagerefnumber{beginappendices}+1}

\textbf{Nombres de pages :} \getpagerefnumber{LastPage} pages dont \arabic{x} d'annexes
\vspace{0.3cm}

\textbf{Système hôte :} \LaTeX
\vspace{1cm}


\textbf{Modifications :} 
\begin{center}
\begin{tabular}{|c|c|c|>{\centering}p{6.5cm}|}
\hline 
EDITION & REVISION & DATE & PAGES MODIFIEES\tabularnewline
\hline
\hline 
1 & 0 & 09/2016 & Création\tabularnewline
\hline 

\end{tabular}
\end{center}
%------------------

%------------------------------------------------------------------------------
% Remerciements
\newevenpage
\chapter*{Remerciements}

Je remercie


%---Résumé (français)---
\begin{abstract}
\thispagestyle{empty}
	\vspace{1cm}

	Ceci est mon résumé bla bla bla
	
	\vspace{1.5cm}
	
	\textbf{Mots clés :} clés, clés, clés
\end{abstract}
%-----------------------


%---Résumé (anglais)---
%\selectlanguage{english}
\begin{abstract}
\thispagestyle{empty}
	\vspace{1cm}
	
	This is my abstract blah blah blah...
	
	\vspace{1.5cm}
	
	\textbf{Key words:} key, key, key
\end{abstract}
%----------------------

\selectlanguage{french}

%---Table des matières, des figures et des tableaux---
\newevenpage
\tableofcontents



\newevenpage
\chapter*{Glossaire et sigles utiles}
\addcontentsline{toc}{chapter}{Glossaire et sigles utiles}

  \begin{acronym}
  \acro{ENSG}{\'Ecole Nationale des Sciences Géographiques}
  \acro{GNSS}{Global Navigation Satellite Systems}
  \acro{GPS}{Global Positionning System}
  \end{acronym}


%---Introduction------------------------------------------------------------------
\newevenpage
\chapter*{Introduction}
  \addcontentsline{toc}{chapter}{Introduction}
  
  \vspace{1.5cm}
	J'introduis

%-------------------------------------------------------------------------------
%\end{comment}

\evenchapter[La nouvelle feuille de style \ensg]{La feuille\newline de style \ensg}

\textit{Voici la version 2 de la feuille de style \ensg. }

\section{Les fichiers}
\begin{itemize}
\item \texttt{themeensg.cls} : contient les personnalisations et macros utiles
\item \texttt{jury.tex} : pour la feuille de présentation du jury
\item le dossier \texttt{images} : il doit contenir toutes les images, il contient déjà le dossier logo avec celui de l'\ensg
\item \texttt{bibliographie.bib} contient la bibliographie
\end{itemize}

\section{Commandes personnalisées}

\begin{itemize}
\item \verb!\newevenpage! : identique à \verb!\newpage! mais en insère une page blanche de façon à débuter la nouvelle page sur un numéro de page impaire.
\item \verb!\evenchapter{titre}! : démarre un nouveau chapitre sur une page impaire,\\ \verb!\evenchapter[titre sommaire]{titre}!   fonctionne aussi mais pas \verb!\evenchapter*{titre}!
\item idem pour \verb!\evenpart{titre}!
\end{itemize}

\section{Fichier source de cette doc}
Ce fichier \texttt{tex} contient toute la structure d'un rapport mais une bonne partie est désactivée car commentée par l'environnement \verb!\begin{comment} ... \end{comment}!

\subsection{Une sous-partie...}

On évitera si possible de faire des sous-sous-parties (\verb!\subsubsection!). Si vous en avez besoin, peut-être faut-il revoir la structures du document...

Au passage voici le code pour appeler une référence de la biblio \cite{globalpositioning} : \verb!\cite{globalpositioning}!

Pour plus d'infos sur la bibli, aller sur \url{http://bertrandmasson.free.fr/index.php?article27/}




%\begin{comment}
%-------------------------------------------------------------------------------
\newevenpage
\chapter*{Conclusion}
  \addcontentsline{toc}{part}{Conclusion}
  \vspace{1.5cm}
Il est l'heure de conclure : bonne nuit !


%-------------------------------------------------------------------------------
% Insertion de la bibliographie
\newevenpage
%\printbibheading
\printbibliography[title={Bibliographie}]
\nocite{*}

\newevenpage
\listoffigures

\newevenpage
\listoftables
%----------------------------------------------------

\newevenpage
\begin{appendices} 
\label{beginappendices}
\annexe[Filtre de Kalman]{Filtre\newline de Kalman}
\label{annexekalman}
Contenu de l'annexe sur Kalman...

\annexe[Moindres carrés]{Moindres carrés}
\label{annexemc}
Contenu de l'annexe sur MC...

\end{appendices} 
%\end{comment}
\end{document}